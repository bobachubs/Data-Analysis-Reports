% Options for packages loaded elsewhere
\PassOptionsToPackage{unicode}{hyperref}
\PassOptionsToPackage{hyphens}{url}
%
\documentclass[
  12pt,
]{article}
\title{36-401 DA Exam 1}
\author{Sarah Li (sarahli)}
\date{October 14, 2022}

\usepackage{amsmath,amssymb}
\usepackage[]{mathpazo}
\usepackage{setspace}
\usepackage{iftex}
\ifPDFTeX
  \usepackage[T1]{fontenc}
  \usepackage[utf8]{inputenc}
  \usepackage{textcomp} % provide euro and other symbols
\else % if luatex or xetex
  \usepackage{unicode-math}
  \defaultfontfeatures{Scale=MatchLowercase}
  \defaultfontfeatures[\rmfamily]{Ligatures=TeX,Scale=1}
\fi
% Use upquote if available, for straight quotes in verbatim environments
\IfFileExists{upquote.sty}{\usepackage{upquote}}{}
\IfFileExists{microtype.sty}{% use microtype if available
  \usepackage[]{microtype}
  \UseMicrotypeSet[protrusion]{basicmath} % disable protrusion for tt fonts
}{}
\makeatletter
\@ifundefined{KOMAClassName}{% if non-KOMA class
  \IfFileExists{parskip.sty}{%
    \usepackage{parskip}
  }{% else
    \setlength{\parindent}{0pt}
    \setlength{\parskip}{6pt plus 2pt minus 1pt}}
}{% if KOMA class
  \KOMAoptions{parskip=half}}
\makeatother
\usepackage{xcolor}
\IfFileExists{xurl.sty}{\usepackage{xurl}}{} % add URL line breaks if available
\IfFileExists{bookmark.sty}{\usepackage{bookmark}}{\usepackage{hyperref}}
\hypersetup{
  pdftitle={36-401 DA Exam 1},
  pdfauthor={Sarah Li (sarahli)},
  hidelinks,
  pdfcreator={LaTeX via pandoc}}
\urlstyle{same} % disable monospaced font for URLs
\usepackage[margin=1in]{geometry}
\usepackage{graphicx}
\makeatletter
\def\maxwidth{\ifdim\Gin@nat@width>\linewidth\linewidth\else\Gin@nat@width\fi}
\def\maxheight{\ifdim\Gin@nat@height>\textheight\textheight\else\Gin@nat@height\fi}
\makeatother
% Scale images if necessary, so that they will not overflow the page
% margins by default, and it is still possible to overwrite the defaults
% using explicit options in \includegraphics[width, height, ...]{}
\setkeys{Gin}{width=\maxwidth,height=\maxheight,keepaspectratio}
% Set default figure placement to htbp
\makeatletter
\def\fps@figure{htbp}
\makeatother
\setlength{\emergencystretch}{3em} % prevent overfull lines
\providecommand{\tightlist}{%
  \setlength{\itemsep}{0pt}\setlength{\parskip}{0pt}}
\setcounter{secnumdepth}{-\maxdimen} % remove section numbering
\ifLuaTeX
  \usepackage{selnolig}  % disable illegal ligatures
\fi

\begin{document}
\maketitle

\setstretch{1.241}
\hypertarget{introduction}{%
\section{Introduction}\label{introduction}}

Airline services constantly struggle to fulfill customer satisfaction
pertaining to the accuracy of flight information. A common complaint
made by passengers is the unexpected delay at the arrival destination.
The industry wants to gain a better insight of factors contributing to
delayed arrivals in order to better provide good services at optimized
prices. Specifically, the Bureau of Transportation Statistics (BTS) has
been tracking every flight's statistics for the last two decades, broken
down into: flight times, taxiing, distance, departure or arrival delay,
etc. \textbf{(1)} In this report, we will specifically analyze the
relationship between flight arrival and departure delay and assess the
linearity between the two variables. Additionally, we will also observe
if their relationship is affected by the presence of weather conditions.

From our analyses, we are hesitant to conclude that there was evidence
of a linear relationship between flight arrival delay and departure
delay, even after data transformation. Additionally, weather doesn't
seem to play a role in the relationship either.

\hypertarget{exploratory-data-analysis-and-initial-modeling}{%
\section{Exploratory Data Analysis and Initial
Modeling}\label{exploratory-data-analysis-and-initial-modeling}}

For this report, we will be parsing in a sample of 4887 flight records
from the ``airlines.csv'' file. We will individually assess the
distributions and important statistics for Departure Delay (the main
predictor variable) and Arrival Delay (the response variable).
\textbf{(2)} From the histograms displayed in Figure 1, we can see the
Departure Delay minutes are unimodal but heavily skewed right. The
minutes range from -29 minutes (negative indicating early departure) to
1099 minutes (delay in departure). The center, or median, is at -1
minutes, indicating that over half the recorded flights departed earlier
than expected. \textbf{(3)} As for Arrival Delays, the distribution of
minutes is also unimodal but heavily skewed right. The minutes range
from -60 minutes (negative indicating early arrival) to 1092 minutes
(delay in arrival). The center is at -2 min, indicating that over half
of the flights arrived early to their destinations.

\begin{figure}
\centering
\includegraphics{sarahli{[}DA1{]}_files/figure-latex/unnamed-chunk-1-1.pdf}
\caption{EDA Analysis of Departure and Arrival Delays}
\end{figure}

Due to heavy skew, we proceed to transform the data by stretching it
outward. Subsequently, we log both Arrival Delay and Departure Delay,
accounting for their minimum values (-60 and -29 minutes, respectively)
by shifting them up by 61 and 30 minutes. This is to ensure that every
data point is included, as the log function only returns finite values
for inputs greater than 0. \textbf{(2)(3)} Shown in Figure 2 below,
though the transformed distributions aren't perfectly symmetric, they
are much more normalized and easy to work with.

\begin{figure}
\centering
\includegraphics{sarahli{[}DA1{]}_files/figure-latex/unnamed-chunk-2-1.pdf}
\caption{EDA Analysis of Departure and Arrival Delays (Transformed)}
\end{figure}

\textbf{(4)} Deciding to work with the log transformations of both
Departure Delay and Arrival Delay, we will perform a bivariate EDA on
the initial modeling of their relationship from the raw data and compare
it with their logged counterparts. From the original scatterplot(left)
in Figure 3, we can see there is an obvious outlier on the top right of
the graph at (1092 minutes, 1099 minutes). We have decided to plot the
relationship between delays and the estimated regression line with and
without this outlier to see if there was high leverage (confirmed there
wasn't and plot omitted).

\begin{figure}
\centering
\includegraphics{sarahli{[}DA1{]}_files/figure-latex/unnamed-chunk-3-1.pdf}
\caption{Scatterplot of Departure and Arrival Delays}
\end{figure}

Additionally, from Figure 3 above, it is visually evident that
Log(Arrival Delay) and Log(Departure Delay) have a positive and
moderately strong linear relationship. Both plots show that many of the
points float above the general trend, potentially leveraging the model
in the upward direction. The points also clutter around values below 100
minutes for both axes. However, there seems to be a few outliers with
comparably lower values of Log(Arrival Delay) and Log(Departure Delay)
to the left half of the scatterplot (\textasciitilde3) that extend the
plot left. From the linear model summary of the log transformation, we
observe a correlation coefficient of .8244, which indicates a relatively
strong correlation between Log(Arrival Delay) and Log(Departure Delay)
(though not necessarily linear).

\hypertarget{diagnostics}{%
\section{Diagnostics}\label{diagnostics}}

\textbf{(5)} The model we have chosen to construct our regression model
is: \textbf{log(Arrival Delay + 61) \textasciitilde{} B0 + B1 x
log(Departure Delay + 30)}. For our sample case,
\(\hat{B_{0}} = 1.8628\) and \(\hat{B_{1}} = 0.6488\).

\textbf{(6)} One major assumption of our model is that the flight
records are random and independent. Since we have taken a sample from
two decades of records, and flight data are not associated, this
assumption can hold. To handle our assumptions of linearity, we want
take a closer look at outliers and points that could influence our
analyses. From the calculation of influential points using Cook's
Distance metric, we can see from Figure 4 that there is 1 main outlier
that exceeded the median of the distribution, determined by comparing
the values to the quantiles of the F distribution with 2 and n-1= 4885
degrees of freedom. of Log(Arrival Delay) and Log(Departure Delay) that
could affect the regression model. The main point was indeed very far
from the main cluster of points but doesn't seem to leverage the data
too much.

\begin{figure}
\centering
\includegraphics{sarahli{[}DA1{]}_files/figure-latex/unnamed-chunk-4-1.pdf}
\caption{Cooks Distance for Transformation}
\end{figure}

\textbf{(6 cont.)} We will proceed by looking at the residuals (errors)
of our model. From Figure 5, we can see that the residuals follow a
similar pattern as the scatterplot. The points clutter around the middle
of the plot and thin out toward the right; the main outlier, labeled
340, also leveraged the residual line upward (though it minorly imapacts
the regression line). If our model assumptions are correct, there should
be no relationship between the residuals and predictor. However, based
on the figure, we can see that the residuals are not evenly distributed
and does not follow a homoskedastic scatter about 0. Even if we were to
run the residuals excluding the outliers taken from cook's distance, the
clustering pattern cannot be ignored. Thus, the residuals violate out
assumptions for a linear model.

\begin{figure}
\centering
\includegraphics{sarahli{[}DA1{]}_files/figure-latex/unnamed-chunk-5-1.pdf}
\caption{Residuals Plot of the Fitted Model}
\end{figure}

To back up this observation, we proceed to plot the Normal Q-Q plot of
the fitted model with and without the outlier of concern at 340. We
observe in Figure 6 that the points in both Q-Q plots do not closely
follow the diagonal line near the ends but still have the same shape and
pattern. In other words, removing the outlier did not change much.

\begin{figure}
\centering
\includegraphics{sarahli{[}DA1{]}_files/figure-latex/unnamed-chunk-6-1.pdf}
\caption{Normal Q-Q Plots of the Fitted Model}
\end{figure}

Thus, after using Cook's Distance to isolate influential data points,
and then confirming through our estimated regression line, residuals,
and Normal Q-Q plot that the single outlier should not leverage our data
in a sample size of 4887, we will keep the outlier in the data for
further interpretation. \textbf{(7)} The assumptions that underlie our
model are reasonable as demonstrated by the figures. There may not be a
linear relationship between Arrival and Departure Delays but there is
arguably an association factored in with other conditions.

\hypertarget{model-inference-and-results}{%
\section{Model Inference and
Results}\label{model-inference-and-results}}

We conclude that there is no simple linear relationship between Arrival
Delays and Departure Delays for flights. That is, an change in unit of
Departure Delay (minutes) can not necessarily predict Arrival Delay
(minutes).

\textbf{(8)} In our regression model, however, there is strong
statistical evidence of there being a relationship between the two
variables. We set \(H_{0}: B_{1} = 0\) and \(H_{0}: B_{1} \neq 0\) as
our null and alternative hypotheses for the simple linear regression
model. \(H_{0}\) tests for no association, a slope of 0, while \(H_{A}\)
tests for an association. From our conducted hypothesis summary, we
conclude a very small and statistically significant p-value \textless{}
0.05. We, in turn, reject \(H_{0}\) and conclude that there is an
association between the predictor (Departure Delay) and response
variable (Arrival Delay) in our transformed log linear model.

\textbf{(9)} Given a mean Departure Delay of 200 minutes, we obtain an
estimated arrival delay of 158.42 minutes. The 90\% confidence for the
expected value of Arrival Delay for all flights given a Departure Delay
of 200 minutes is (153.962, 162.97). We are 90\% confident that the true
population mean of Arrival Delays for the population of Departure Delays
of 200 minutes is between 158.42 minutes and 162.9701 minutes.

\begin{verbatim}
##        fit      lwr      upr
## 1 158.4199 153.9622 162.9701
\end{verbatim}

\textbf{(10)} As stated in our introduction, we have to assess whether
weather conditions affect the relationship between Departure and Arrival
delays. We decided to study the flight delays when weather conditions
were present and not present, separately.

\begin{figure}
\centering
\includegraphics{sarahli{[}DA1{]}_files/figure-latex/unnamed-chunk-8-1.pdf}
\caption{Transformed Delays w/ and w/o Weather Conditions}
\end{figure}

\textbf{(10 cont.)} In Figure 7, we observe that the relationship
between Log(Arrival Delays) and Log(Departure Delays) both with and
without weather conditions is positive and has a moderately strong
linear relationship. After implicitly running the summaries of both
models, we observe very small p-values (\textless{} 0.05 alpha),
indicating a statistically significant relationship for our separately
fitted models conditioning on weather disruptions. Subsetting on flights
with weather conditions, we are 90\% the true slope of the fitted model
predicting Arrival Delay from Departure Delay is within (0.552, 0.704).
For flights without weather conditions, we are 90\% confident the true
slope of the fitted model is within (0.63, 0.6516). However, since these
confidence intervals overlap, we cannot conclude a significant
difference in the fitted models between the relationship of Arrival and
Departure delays given weather conditions. In other words, the presence
of weather disruptions doesn't seem to affect a flight's relationship
between Arrival and Departure delays for our model.

\begin{verbatim}
##                   5 %      95 %
## (Intercept) 1.7296093 2.4240025
## LogDepDelay 0.5520865 0.7043341
## Weather            NA        NA
\end{verbatim}

\begin{verbatim}
##                   5 %      95 %
## (Intercept) 1.8507308 1.9275912
## LogDepDelay 0.6299204 0.6516073
## Weather            NA        NA
\end{verbatim}

\hypertarget{conclusion-and-discussion}{%
\section{Conclusion and Discussion}\label{conclusion-and-discussion}}

\textbf{(11)} After conducting a thorough analysis of the relationship
between Departure Delays and Arrival Delays for flights, we decisively
conclude that there does not exist a linear relationship between the two
variables. The reasoning follows from the violations of a simple linear
regression. Even though we assumed our records were i.i.d and
transformed both the predictor and response variables to have roughly
normal distributions, the scatterplot and residuals vs fitted values
demonstrate that the erroros are not normal and uncorrelated. There is
evidence of heteroskedasticity about 0, with values fanning starting
from low values of both Arrival and Departure Delays. Factoring the
presence of weather conditions, we also cannot conclude a significant
difference among flights with and without them. However, given
relatively high correlation coefficients for th fitted transformed
models, we can conduct further analyses and testing for the possibility
of another type of association (i.e.~Multiple linear regression)

\end{document}
